\documentclass[12pt]{article}

\usepackage[margin=1.0in]{geometry}
\usepackage[USenglish]{isodate}
\usepackage{tabularx}
\usepackage[usenames, dvipsnames]{color}
\usepackage{pbox}
\usepackage{paralist}

\vspace{5mm}
\setlength{\tabcolsep}{20pt}
\renewcommand{\arraystretch}{1.5}

%------------------------------------------------------------------------------
% Paper Title
%------------------------------------------------------------------------------

\title{Software Development Plan}
\author{Dunham, Martinear, Richardson, and Traglia}
\date{\today}

\begin{document}

\maketitle
\newpage
\tableofcontents
\newpage

%------------------------------------------------------------------------------
% Project Overview
%------------------------------------------------------------------------------

\section{Project Overview}

\subsection{Description}
PoissonBarrel, is a statistical analysis software program which analyzes a set
of data input by the user, makes calculations based on the statistical measure
chosen by the user, and displays the results in a graph of the user's choice.
For example, after entering the data, the user may choose a statistical measure
for analysis (i.e. mean, median, mode, or standard deviation) and then choose a
type of graph (i.e. vertical bar graph, X-Y Plot graph, table) to display the
data. In short, PoissonBarrel functions similarly to Microsoft Excel. \\

PoissonBarrel is named after the French mathematician Siméon Denis Poisson, a
pioneer in modern probability theory and the namesake of the Poisson
distribution. Poisson also means ``fish'' in French, and since this application
is a ``barrel of'' statistical analyses.

\subsection{Features}
\begin{itemize}
\item Allows the user to enter data for analysis
\item Provides a list of statistical measures to analysis the data
\item Provides a list of graph types to represent the data
\item Displays graphical representation of analysis results
\item Creates a text file summarizing analysis results
\item Allows user to load, save, edit and delete analysis results
\end{itemize}

\subsection{Statistical Options}
The following is a list of statistical functionality components which can be
performed on the table data displayed by the application. To minimize code
revision as new statistical components are added, the application will utilize
the Strategy pattern. Meaning that all Statistical Component classes will be
sub-classed from an abstract class containing a single method:
\textit{compute()}. Each subclass will be required to provide its own
implementation of \textit{compute()}.

\begin{center}
\begin{tabularx}{\textwidth}{|X||p{72mm}|}
\hline \textbf{Statistical Measure Types} & \\
\hline Mean & Median \\
\hline Mode & Standard Deviation \\
\hline Variance & Coefficient of Variance \\
\hline Standard Deviation & Percentiles \\
\hline Probability Distribution & Binomial Distribution \\
\hline Least Square Line & Chi Square \\
\hline Correlation Coefficient & Sign Test \\
\hline Rank Sum Test & Spearman Rank Correlation Coefficient \\
\hline
\end{tabularx}
\end{center}

There are also multiple supported chart/diagram types. These types are listed
and described in the following table:

\begin{center}
\begin{tabularx}{\textwidth}{|X|}
\hline \textbf{Chart/Diagram Types} \\
\hline Horizontal/Vertical Bar Chart \\
\hline Pie Chart \\
\hline Normal Distribution Curve \\
\hline X-Y Graph \\
\hline
\end{tabularx}
\end{center}

%------------------------------------------------------------------------------
% Project Components
%------------------------------------------------------------------------------

\section{Components}
\subsection{File I/O Components}
There are three components that deal with file operations. These are listed and
described in the following table:

\begin{center}
\begin{tabularx}{\textwidth}{|X|p{8cm}|}
	\hline \textbf{File I/O Components} & \textbf{Description} \\
	\hline Comma/Tab Delimited Files & The application contains a module that
		enable it to read, validate, and write CSV files that can be viewed in
		Microsoft Excel. \\
	\hline Summary Text Files & The application writes text files that
		summarize the statistical analysis performed by the user on the
		document’s data. \\
	\hline JPEG Image Files & The application has a module that allows the user
		to create JPEG images of any chart or graph that the application
		generates. \\
	\hline
\end{tabularx}
\end{center}

\subsection{Graphical User Interface}
The largest component is the graphical interface. There are multiple
sub-components that collectively make up the graphical interface. These
components are listed and described in the following table:

\begin{center}
\begin{tabularx}{\textwidth}{|p{3cm}|X|}
	\hline \textbf{Graphic User \newline Interface Components} &
		\textbf{Description} \\
	\hline Main Window & The application has a main window widget that allows
		the user to interact with the displayed data through mouse and keyboard
		input. Additionally, the main window contains a menu bar allowing the
		user to open, close, and save documents, as well as providing access to
		the Chart/Diagram Creation Wizard. \\
	\hline Spreadsheet \newline Table & The application has a spreadsheet table
		widget contained within the main window's client area that allows the
		user to create, view, modify, and analyze statistical data. \\
	\hline Open/Save File \newline Dialog Boxes & The application allows the
		user to open and save files through the modal dialog boxes that provide
		graphical user interface to the operating system's file management
		system. \\
	\hline Chart/Diagram \newline Creation Wizard Dialog & The application
		provides a modeless dialog box that allows for user friendly creation
		of charts and diagrams. \\
	\hline Chart/Diagram Windows & The application generates and displays
		charts and diagrams in a window separate from that of the of the main
		window. Charts and diagrams are generated by the data contained within
		the spreadsheet table. The window shall have an option that allows the
		user to create a JPEG image file of the displayed chart. \\
	\hline
\end{tabularx}
\end{center}

%------------------------------------------------------------------------------
% Project Schedule
%------------------------------------------------------------------------------

\section{Project Schedule}
\subsection{Deliverables}
There are four project deliverables, as defined in the table below, that will
be given to the customer. For each deliverable, there will be given a
presentation of it to the customer.

\begin{center}
\begin{tabularx}{\textwidth}{|X|l|l|}
	\hline \textbf{Deliverable} & \textbf{Deadline} \\
	\hline Project Requirements and Backlog & \printdate{2016-9-27} \\ 
	\hline Architectural Design & \printdate{2016-10-18} \\
	\hline User Interface Design & \printdate{2016-11-3} \\
	\hline Product Delivery & \printdate{2016-11-29} \\
	\hline
\end{tabularx}
\end{center}

\subsection{Milestones}
For this project, there are loosely defined milestones our team wishes to
complete by certain deadlines. These milestones will encourage continuous
development and mark certain stages of the project.

\begin{center}
\begin{tabularx}{\textwidth}{|X|l|l|}
	\hline \textbf{Milestone} & \textbf{Deadline} \\
	\hline Create Proof-of-Concept GUI & \printdate{2016-9-13} \\
	\hline Develop Statistical Algorithms and Spreadsheet UI &
		\printdate{2016-9-29} \\
	\hline Begin Testing on Statistical Functionality And Development on Chart
		Generation & \printdate{2016-10-20} \\
	\hline Continue Statistical Functionality and GUI development/testing,
		Develop non-required GUI features (Undo/Redo, Copy \& Path, Drag \&
		Drop, AutoFill) & \printdate{2016-11-8} \\
	\hline Complete All Testing and Release Product & \printdate{2016-11-22} \\
	\hline
\end{tabularx}
\end{center}

%------------------------------------------------------------------------------
% Team Members
%------------------------------------------------------------------------------

\section{Team Members}
The team consists of four members. Each team member is assigned one unique
position, in addition to the following positions: \textit{Requirements
Analyst}, \textit{Designer}, \textit{Programmer}, \textit{Test Designer}, and
\textit{Test}.

\subsection*{Richard Dunham\hfill\textit{Software Lead}}
Richard Dunham is a Computer Science major with a minor in Mathematics. As an
autodidactic programmer, Richard is well suited for Agile team development as
he is proficient in designing flexible, customizable software through the use
of the Object-Oriented paradigm. Additionally, Richard provides unique assets
to the team with a background in Computational Geometry, Graph Theory, and
Computer Graphics. When he is not coding, or pouring over API documentation,
Richard enjoys mystery novels and the great outdoors.

\subsection*{Chelsie Martinear\hfill\textit{Customer Liaison}}
Chelsie is a computer science and mathematical sciences major who has an
extensive background in programming, numerical computation, and technical
writing. Additionally, she has over five years of experience in customer
relations. As a valuable member of this team, she is dedicated to the
development of PoissonBarrel and is determined to put her best effort into
delivering and presenting a polished product to the customer. 

\subsection*{Joseph Richardson\hfill\textit{Scrum Master}}
Joseph is returning to computer science after completing bachelor's and
master's degrees in history, with additional background in classical languages
and English literature. His liberal arts roots bring versatility and unique
perspectives and skills to his work in information technology. Some of his
fortes in computers include regular expressions, data mining, databases, and
scripting languages.

\subsection*{Justin Traglia\hfill\textit{Product Owner}}
With over four years of software engineering experience, Justin is qualified to
work on this project. He has experience working with small groups, agile
programming development, version control, and documentation. With a completed
minor in mathematics, Justin has the ability to help develop many of the
algorithms for this project, and should be a valuable member of the team.

%------------------------------------------------------------------------------
% Risk Management
%------------------------------------------------------------------------------

\section{Risk Management}
As with all projects there are risks. Our team has conceived of eight possible
risks. There will be a description of each risk, along with a plan of what to
do if it were to happen.

\subsection*{Death or Dismemberment of Team Member(s)}
\begin{tabular}{ l p{10cm} }
	\textbf{Chance} & \textit{Minimal} \\
	\textbf{Severity} & \textit{Catastrophic} \\
	\textbf{Description} & If any of the group members were to die or lose body
		parts essential to working on this project. \\
	\textbf{Contingency} & Distribute the team member(s) share of the project
		to the rest of the team. If all team members were to become unable to
		work, this project would be the least of our concerns. \\
\end{tabular}

\subsection*{Negligence of Group Member(s)}
\begin{tabular}{ l p{10cm} }
	\textbf{Chance} & \textit{Possible} \\
	\textbf{Severity} & \textit{Marginal} \\
	\textbf{Description} & If any of the team member(s) were to ``slack off''
		and not do their part of the project. \\
	\textbf{Contingency} & We would contact the customer and inform them of the
		issue. If the issue is not solved, we would fire the team member(s),
		distribute their share of the project to the rest of the team, and
		continue working on the project. \\
\end{tabular}

\subsection*{Unexpected Obligations Outside of Project}
\begin{tabular}{ l p{10cm} }
	\textbf{Chance} & \textit{Possible} \\
	\textbf{Severity} & \textit{Marginal} \\
	\textbf{Description} & Since every member of the team is a full-time
		student, we all have obligations outside of this project. Obligations
		such as exams and other class projects might temporarily prevent a team
		member from working on their tasks. \\
	\textbf{Contingency} & If this were to happen to a team member, their tasks
		would be temporarily assigned to team members without such obligations.
		Once free, the aforementioned tasks would be re-assigned back to the
		original team member. \\
\end{tabular}

\subsection*{Team Member(s) Unavailable for Meetings}
\begin{tabular}{ l p{10cm} }
	\textbf{Chance} & \textit{Very Possible} \\
	\textbf{Severity} & \textit{Marginal} \\
	\textbf{Description} & Due to the team member's busy schedules, there might
		be instances when certain members are unable to attend group meetings.
		At times, the entire team may be unable to gather for a meeting. \\
		\textbf{Contingency} & If the team is unable to meet in person, we will
		have online meetings using \textit{Slack} and/or \textit{Skype}.\\
\end{tabular}

\subsection*{Poor Organization/Lack of Leadership}
\begin{tabular}{ l p{10cm} }
	\textbf{Chance} & \textit{Minimal} \\
	\textbf{Severity} & \textit{Serious} \\
	\textbf{Description} & If tasks are not properly broken up and reasonably
		assigned, the project may become unorganized and as a result become
		difficult to develop. Similarly, the team may lack direction if there
		is no leader. \\
	\textbf{Contingency} & If the project becomes unorganized, we will
		temporarily stop development to re-organized the project. For
		direction, the team must work together to create new tasks and move the
		project forward. \\
\end{tabular}

\subsection*{Unable to Understand Math and Algorithms}
\begin{tabular}{ l p{10cm} }
	\textbf{Chance} & \textit{Possible} \\
	\textbf{Severity} & \textit{Serious} \\
	\textbf{Description} & Some of the statistical measures are unknown to the
		team. If we are unable to understand these measures, we will not be
		able to program them. \\
	\textbf{Contingency} & If after research we still do not understand
		anything, we will contact a mathematics professor and ask for help. \\
\end{tabular}

\subsection*{Challenging GUI Framework}
\begin{tabular}{ l p{10cm} }
	\textbf{Chance} & \textit{Very Possible} \\
	\textbf{Severity} & \textit{Critical} \\
	\textbf{Description} & The project language, Python, does not have many
		good GUI frameworks. As a result, creating the graphical interface
		might be difficult. \\
	\textbf{Contingency} & Spend more time researching how to use the GUI
		framework. Attempt to overcome it's faults. \\
\end{tabular}

\end{document}
