\documentclass[12pt]{article}

\usepackage[margin=1.0in]{geometry}
\usepackage[USenglish]{isodate}
\usepackage{tabularx}
\usepackage[usenames, dvipsnames]{color}

\vspace{5mm}
\setlength{\tabcolsep}{20pt}
\renewcommand{\arraystretch}{1.5}

%------------------------------------------------------------------------------
% Paper Title
%------------------------------------------------------------------------------

\title{Software Development Plan}
\author{Dunham, Martinear, Richardson, and Traglia}
\date{\today}

\begin{document}

\maketitle
\newpage
\tableofcontents
\newpage

%------------------------------------------------------------------------------
% Project Overview
%------------------------------------------------------------------------------

\section{Project Overview}

\subsection{Description}

\subsection{Objectives}

%------------------------------------------------------------------------------
% Project Schedule
%------------------------------------------------------------------------------

\section{Project Schedule}

\subsection{Components}

\subsubsection{GUI}

\subsection{Deliverables}

There are four project deliverables, as defined in the table below, that will
be given to the customer. For each deliverable, there will be given a
presentation of it to the customer.

\begin{center}
\begin{tabularx}{\textwidth}{|X|l|l|}
    \hline \textbf{Deliverable} & \textbf{Deadline} \\
    \hline Project Requirements and Backlog & \printdate{2016-9-27} \\ 
    \hline Architectural Design & \printdate{2016-10-18} \\
    \hline User Interface Design & \printdate{2016-11-3} \\
    \hline Product Delivery & \printdate{2016-11-29} \\
    \hline
\end{tabularx}
\end{center}

\subsection{Milestones}

For this project, there are loosely defined milestones our team wishes to
complete by certain deadlines. These milestones will encourage continuous
development and mark certain stages of the project.

\begin{center}
\begin{tabularx}{\textwidth}{|X|l|l|}
    \hline \textbf{Deliverable} & \textbf{Deadline} \\
    \hline Create Proof-of-Concept GUI & \printdate{2016-1-1} \\ 
    \hline Complete Mathematical Functions & \printdate{2016-1-1} \\
    \hline Complete Presentation of Mathematical Functions & \printdate{2016-1-1} \\
    \hline Compete GUI & \printdate{2016-1-1} \\
    \hline Complete Import and Parsing of Data & \printdate{2016-1-1} \\
    \hline Complete Final Project & \printdate{2016-1-1} \\
    \hline Complete Verification and Validation of Project & \printdate{2016-1-1} \\
    \hline
\end{tabularx}
\end{center}

%------------------------------------------------------------------------------
% Team Members
%------------------------------------------------------------------------------

\section{Team Members}

The team consists of four members. Each team member is assigned one unique
position, in addition to the following positions: \textit{Requirements
Analyst}, \textit{Designer}, \textit{Programmer}, \textit{Test Designer}, and
\textit{Test}.

\subsection{Richard Dunham\hfill\textit{Software Lead}}

\textit{Please write me!}

\subsection{Chelsie Martinear\hfill\textit{Customer Liaison}}

\textit{Please write me!}

\subsection{Joeseph Richardson\hfill\textit{Scrum Master}}

\textit{Please write me!}

\subsection{Justin Traglia\hfill\textit{Product Owner}}

With over four years of software engineering experience, Justin is qualified to
work on this project. He has experience working with small groups, agile
programming development, version control, and documentation. With a completed
minor in mathematics, Justin has the ability to help develop many of the
algorithms for this project, and should be a valuable member of the team.

%------------------------------------------------------------------------------
% Risk Management
%------------------------------------------------------------------------------

\section{Risk Management}

For this project, there are the following risks:

%
% I'm not sure if a table is appropriate for this, it might be better to break
% this up into sections. Or maybe even a list, so that you can go into more
% detail about each risk.
%

\begin{center}
\begin{tabularx}{\textwidth}{|X|l|l|}
    \hline \textbf{Risk} & \textbf{Severity} \\
    \hline We all die & \color{Red}Catastrophic \\
    \hline We each lose a hand & \color{Orange}Critical \\
    \hline Senioritis & \color{Yellow}Marginal \\
    \hline We get sick once & \color{Green}Negligible \\
    \hline
\end{tabularx}
\end{center}

\end{document}
